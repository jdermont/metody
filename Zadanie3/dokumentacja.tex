%& --translate-file=cp1250pl
\documentclass[12pt,a4paper]{article}
\usepackage[left=0.75in,right=0.75in,top=0.75in,bottom=0.75in,dvips]{geometry}


\usepackage{polski}
\usepackage[OT4]{fontenc}
\usepackage{amsthm,amsmath,amsfonts,amssymb}

\usepackage{verbatim}

\usepackage{color}
\usepackage{alltt}
\usepackage{enumerate}
\usepackage{url}
\usepackage{listings}
\lstset{language=C++,
tabsize=1,
basicstyle=\ttfamily\footnotesize,
frame=shadowbox,
breaklines=true,
showstringspaces=false}

\author{Marcin Horoszko, Rados�aw G�ombiowski, Jacek Dermont}
\title{Dokumentacja do zadania 3.14}

\newtheorem{tw}{Cz��}[section]
\theoremstyle{definition}
\newtheorem{defi}[tw]{}

\begin{document}

\maketitle

\section{Zadanie 3.14}
Zagadnienie r�niczkowe $xy' = (y - 2)y - x^4, y(1) = 3$ rozwi�za� na przedziale $[1,3]$ metod� Eulera oraz udoskonalon� metod� Eulera, zwan� metod� Heuna.
Wyniki por�wna� z rozwi�zaniem dok�adnym $y(x) = x^2 + 2$.

\section{Podstawowe poj�cia (TODO)}
\begin{defi}[R�wnanie r�niczkowe]
...
\end{defi}
\begin{defi}[Zagadnienie Cauchy'ego]
Zagadnienie polegaj�ce na znalezieniu konkretnej funkcji spe�niaj�cej dane r�wnanie r�niczkowe i warunek pocz�tkowy. W przypadku r�wnania stopnia
pierwszego, warunkiem pocz�tkowym b�dzie punkt, przez kt�ry powinien przechodzi� wykres szukanej funkcji. W przypadku r�wnania stopnia drugiego, zagadnienie
pocz�tkowe zawiera� b�dzie dodatkowo warto�� pierwszej pochodnej w danym punkcie i analogicznie, w przypadku r�wna� wy�szego stopnia. \\
TODO: poda� przyk�ad
\end{defi}
\begin{defi}[Metoda Eulera]
...
\end{defi}
\begin{defi}[Udoskonalona metoda Eulera - metoda Heuna]
...
\end{defi}
\end{document}